% vim:encoding=utf8 ft=tex sts=2 sw=2 et:

\documentclass{classrep}
\usepackage[utf8]{inputenc}
\usepackage{fixltx2e}
\usepackage{url}
\usepackage{graphicx}
\usepackage{siunitx}

\studycycle{Informatyka, studia niestacjonarne, inż I st.}
\coursesemester{VI}

\coursename{Sztuczna inteligencja i systemy ekspertowe}
\courseyear{2013/2014}

\courseteacher{dr inż. Krzysztof Lichy}
\coursegroup{sobota, 15:30}

\author{
  \studentinfo{Łukasz Ochmański}{183566} \and
  \studentinfo{Przemysław Szwajkowski}{173524}
}

\title{Zadanie 2 - FuzzyLogic}
\svnurl{https://sise-lukasz-ochmanski.googlecode.com/svn/trunk/02}

\begin{document}
\maketitle


\section{Wprowadzenie}
Celem niniejszego zadania jest napisanie aplikacji, który zasymuluje parkujący samochód. Program ma wykorzystawać logikę rozmytą i ma być napisany przy użyciu języka FCL (Fuzzy Control Language).

\section{Uruchamianie programu}
Program można uruchomić z lini poleceń w systemie z zainstalowaną wirtualną maszyną Java'y wersji 8 lub nowszej. Program wykorzystuje biblioteki JavaFX 2.2, które są wymagane do poprawnego działania programu.

Przed uruchomieniem należy spakować projekt wraz z bibliotekami do formatu *.jar. Należy pamiętać o zachowaniu folderu o nazwie: resources, który zawiera obrazy oraz style css niezbędne do otwarcia inferejsu graficznego aplikacji. Wszystkie zasoby powinny mieścić się w jednym folderze, aby aplikacja działała poprawnie.

\begin{figure}[ht]
\centering
			\includegraphics[scale=0.60]{pictures/Obraz01.png}
	\caption{Prototyp aplikacji}
	\label{fig:Prototyp aplikacji}
\end{figure}

\begin{figure}[ht]
\centering
			\includegraphics[scale=0.60]{pictures/Obraz13.png}
	\caption{Obszary determinujace kierunek jazdy}
	\label{fig:Obszary determinujace kierunek jazdy}
\end{figure}

\begin{figure}[ht]
\centering
			\includegraphics[scale=0.60]{pictures/Obraz14.png}
	\caption{Obszary determinujace stopien skretu}
	\label{fig:Obszary determinujace stopien skretu}
\end{figure}

\begin{figure}[ht]
\centering
			\includegraphics[scale=0.80]{pictures/Obraz02.png}
	\caption{Wykres przynaleznosci}
	\label{fig:Wykres przynaleznosci}
\end{figure}

\begin{figure}[ht]
\centering
			\includegraphics[scale=0.8]{pictures/Obraz03.png}
	\caption{Wykres przynaleznosci}
	\label{fig:Wykres przynaleznosci}
\end{figure}

\begin{figure}[ht]
\centering
			\includegraphics[scale=0.8]{pictures/Obraz04.png}
	\caption{Wykres przynaleznosci}
	\label{fig:Wykres przynaleznosci}
\end{figure}

\begin{figure}[ht]
\centering
			\includegraphics[scale=0.8]{pictures/Obraz05.png}
	\caption{Wykres przynaleznosci}
	\label{fig:Wykres przynaleznosci}
\end{figure}

\begin{figure}[ht]
\centering
			\includegraphics[scale=0.8]{pictures/Obraz06.png}
	\caption{Wykres przynaleznosci}
	\label{fig:Wykres przynaleznosci}
\end{figure}

\begin{figure}[ht]
\centering
			\includegraphics[scale=0.8]{pictures/Obraz07.png}
	\caption{Wykres przynaleznosci}
	\label{fig:Wykres przynaleznosci}
\end{figure}

\begin{figure}[ht]
\centering
			\includegraphics[scale=0.8]{pictures/Obraz08.png}
	\caption{Wykres przynaleznosci}
	\label{fig:Wykres przynaleznosci}
\end{figure}

\begin{figure}[ht]
\centering
			\includegraphics[scale=0.8]{pictures/Obraz09.png}
	\caption{Wykres przynaleznosci}
	\label{fig:Wykres przynaleznosci}
\end{figure}

\begin{figure}[ht]
\centering
			\includegraphics[scale=0.60]{pictures/Obraz10.png}
	\caption{Dzialajacy model pojazdu}
	\label{fig:Dzialajacy model pojazdu}
\end{figure}

\begin{figure}[ht]
\centering
			\includegraphics[scale=0.60]{pictures/Obraz11.png}
	\caption{Obraz przedstawiajacy pojazd zmierzajacy do celu}
	\label{fig:Obraz przedstawiajacy pojazd zmierzajacy do celu}
\end{figure}

\begin{figure}[ht]
\centering
			\includegraphics[scale=0.60]{pictures/Obraz12.png}
	\caption{Przyklad zawracania}
	\label{fig:Przyklad zawracania}
\end{figure}

\clearpage
\section{Teoria}
Sterownik składa sie z czterech elementów:
\subsection{Baza reguł}
Bazę reguł stanowi zbiór rozmytych reguł postaci:
IF (x\textsubscript{1} jest A\textsubscript{1}) AND ... AND (x\textsubscript{n} jest A\textsubscript{n})
THEN (y\textsubscript{1} jest B\textsubscript{1}) AND ... AND (y\textsubscript{m} jest B\textsubscript{m})

gdzie A\textsubscript{i}, B\textsubscript{j}, i = 1, ..., n, j = 1, ..., m – sa zbiorami rozmytymi, x\textsubscript{i} –
zmiennymi wejściowymi, yj – zmiennymi wyjściowymi.

\subsection{Blok rozmywania}
Ponieważ system sterowania z logika rozmytą operuje na zbiorach rozmytych,
dlatego konkretne wartości sygnału wejściowego podlegają operacji
rozmywania, w wyniku której zostają one odwzorowane w zbiór rozmyty.
\subsection{Blok wnioskowania}
Na podstawie zbioru reguł rozmytych w oparciu o uogólnione reguły
wnioskowania znajdujemy odpowiedni zbiór rozmyty bedący wnioskiem
powstałym w oparciu o podane przesłanki.
\subsection{Blok wyostrzania}
Wielkością wyjściowa bloku wnioskowania jest blok rozmyty. Zbiór ten
należy odwzorować w jedną wartość, która będzie poszukiwanym stanem
sterującym.

\clearpage

\section{Wnioski}
  Niestety nie udało nam się doprowadzić programu do stanu, w którym prawidłowo parkowałby na wstecznym biegu. Okazało się to dość skomplikowanym zadaniem. Samochód jest jednak wyposażony w pewną inteligencję, ponieważ potrafi rozpoznać niebezpieczną odległość od ściany, oraz zbyt bliską odległość od innych pojazdów. Nie powinien wjechać w otaczające go samochody. Wie, że gdy jest zbyt blisko, nie uda się wjechać w wolne miejsce parkingowe pod ostrym kątem. Należy wtedy cofnąć i wjechać z dużej odległości.



\begin{thebibliography}{0}
  \bibitem{l2short} T. Oetiker, H. Partl, I. Hyna, E. Schlegl.
    \textsl{Nie za krótkie wprowadzenie do systemu \LaTeX2e}, 2007, dostępny
    online.
\end{thebibliography}
\end{document}